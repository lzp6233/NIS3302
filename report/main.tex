%!TeX program = xelatex
\documentclass[12pt,hyperref,a4paper,UTF8]{ctexart}
\usepackage{SJTUReport}

%%-------------------------------正文开始---------------------------%%
\begin{document}

%%-----------------------封面--------------------%%
\cover

%%------------------摘要-------------%%
\begin{abstract}

在此填写摘要内容

\end{abstract}

\thispagestyle{empty} % 首页不显示页码

%%--------------------------目录页------------------------%%
\newpage
\tableofcontents

%%------------------------正文页从这里开始-------------------%
\newpage
\section{需求分析}
% 具体内容涉及(但不局限于)以下方面:
% a)	项目需求分析,即描述本项目需要解决什么样的安全问题
% b)	项目所要完成的功能目标

\section{总体设计}
% 主要内容包括(但不局限于)以下方面:
% 	总体结构图:结构图中需要体现出所涉及的每个模块,以及模块之间的信息交互。
% 	项目的模块划分,并逐个介绍每个模块:
% 	模块名称
% 	模块的功能
% 	模块的输入,或依赖的信息
% 	模块的输出

\section{详细设计}
% 每个模块可对应一节,这里假定含三个模块,每一节包括(但不局限于)以下内容:
% 	模块概述:功能、输入、输出等
% 	模块的数据处理流程:以流程图的形式给出,也可以分散在每个函数中
% 	主要数据结构的设计
% 	对主要的处理函数逐个进行设计,每个函数的详细设计包括
% 	函数名,输入参数,输出类型
% 	函数功能
% 	依赖哪些外部函数
% 	函数处理流程图

\section{系统实现与测试}
% 内容包括(但不局限于)以下内容:
% 	实现环境、开发工具等介绍
% 	测试环境搭建
% 	具体的测试方法
% 	测试流程:最好有类似流程图的方式体现
% 	具体进行了哪些测试,每个测试观察到什么样的现象、效果,有何结论
% 	所有测试结束后,得出了哪些结论

\section{项目总结}

\section{分工}

%%可选择这里也放一个标题
%\begin{center}
%    \title{ \Huge \textbf{{标题}}}
%\end{center}



%%----------- 参考文献 -------------------%%
%在reference.bib文件中填写参考文献,此处自动生成

\reference


\end{document}