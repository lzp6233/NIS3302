%!TeX program = xelatex
\documentclass[12pt,hyperref,a4paper,UTF8]{ctexart}
\usepackage{SJTUReport}
\usepackage{tikz}
\usepackage{geometry}
\usepackage{graphicx}
\usepackage{float}

%%-------------------------------正文开始---------------------------%%
\begin{document}

%%-----------------------封面--------------------%%
\cover

%%------------------摘要-------------%%
\begin{abstract}

本项目设计并实现了一个基于Web的网络端口扫描工具,旨在为网络安全测试提供直观、高效的端口扫描解决方案。系统采用C++后端和HTML5前端相结合的架构,支持多种扫描方式包括TCP SYN扫描、TCP Connect扫描、UDP扫描以及ICMP ping检测。

系统主要功能包括:多线程高速端口扫描、实时扫描进度显示、扫描结果可视化展示、扫描历史记录管理、以及基于开放端口的网络安全风险评估。后端采用httplib库构建RESTful API服务,前端使用现代Web技术实现响应式用户界面。

通过实际测试验证,系统能够准确识别目标主机的开放端口,扫描速度达到每秒数百个端口,具有良好的稳定性和用户体验。该工具为网络安全专业人员提供了一个功能完善、易于使用的端口扫描解决方案。

\end{abstract}

\thispagestyle{empty} % 首页不显示页码

%%--------------------------目录页------------------------%%
\newpage
\tableofcontents

%%------------------------正文页从这里开始-------------------%
\newpage
\section{需求分析}

\subsection{项目背景}
随着网络技术的快速发展,网络安全问题日益突出。端口扫描作为网络安全评估的基础工具,对于发现网络漏洞、评估系统安全性具有重要意义。传统的命令行端口扫描工具虽然功能强大,但缺乏直观的用户界面,对于非专业用户来说使用门槛较高。

\subsection{项目需求}
本项目旨在开发一个基于Web的网络端口扫描工具,主要解决以下问题:

\begin{enumerate}
    \item \textbf{用户友好性}:提供直观的图形用户界面,降低使用门槛
    \item \textbf{功能完整性}:支持多种扫描方式,满足不同场景需求
    \item \textbf{性能优化}:采用多线程技术提高扫描效率
    \item \textbf{结果可视化}:以图表形式展示扫描结果,便于分析
    \item \textbf{历史管理}:保存扫描历史,支持结果对比和趋势分析
\end{enumerate}

\subsection{功能目标}
系统需要实现以下核心功能:

\begin{itemize}
    \item \textbf{ICMP扫描}:检测目标主机是否可达
    \item \textbf{TCP SYN扫描}:快速识别开放端口,避免建立完整连接
    \item \textbf{TCP Connect扫描}:建立完整TCP连接进行端口检测
    \item \textbf{UDP扫描}:检测UDP端口状态
    \item \textbf{多线程扫描}:支持自定义线程数,提高扫描效率
    \item \textbf{实时进度显示}:显示扫描进度和状态
    \item \textbf{结果可视化}:以表格和图表形式展示扫描结果
    \item \textbf{历史记录}:保存和管理扫描历史
    \item \textbf{结果导出}:支持扫描结果导出功能
\end{itemize}

\section{总体设计}

\subsection{系统架构}
系统采用前后端分离的架构设计,如图\ref{fig:architecture}所示:

\begin{figure}[h]
\centering
\begin{tikzpicture} [
    box/.style={rectangle, draw, rounded corners, minimum width=2cm, minimum height=1cm, align=center},
    arrow/.style={->, thick}
]
    % 前端层
    \node[box, fill=blue!20] (web) at (0,3.5) {Web前端\\HTML5/CSS/JS};
    
    % API层
    \node[box, fill=green!20] (api) at (0,1.5) {RESTful API\\httplib};
    
    % 后端服务层
    \node[box, fill=yellow!20] (backend) at (0,0) {C++后端服务};
    
    % 功能模块层
    \node[box, fill=orange!20] (icmp) at (-3.5,-2) {ICMP模块\\ping.h/cpp};
    \node[box, fill=orange!20] (port) at (0,-2) {端口扫描模块\\PortScanner.h/cpp};
    \node[box, fill=orange!20] (network) at (3.5,-2) {网络模块\\network.h/cpp};

    % 连接(加长箭头)
    \draw[arrow] (web) -- ++(0,-0.9) -| (api);
    \draw[arrow] (api) -- ++(0,-0.9) -| (backend);
    \draw[arrow] (backend) -- ++(-0.5,-0.9) -| (icmp);
    \draw[arrow] (backend) -- ++(0,-0.9) -| (port);
    \draw[arrow] (backend) -- ++(0.5,-0.9) -| (network);
\end{tikzpicture}
\caption{系统总体架构图}
\label{fig:architecture}
\end{figure}

\subsection{模块划分}

\subsubsection{Web前端模块}
\begin{itemize}
    \item \textbf{功能}:提供用户界面,处理用户交互,展示扫描结果
    \item \textbf{输入}:用户配置参数,后端API响应
    \item \textbf{输出}:用户界面更新,扫描请求发送
    \item \textbf{依赖}:现代Web浏览器,后端API服务
\end{itemize}

\subsubsection{RESTful API模块}
\begin{itemize}
    \item \textbf{功能}:提供HTTP API接口,处理前端请求
    \item \textbf{输入}:HTTP请求(GET/POST)
    \item \textbf{输出}:JSON格式响应数据
    \item \textbf{依赖}:httplib库,后端功能模块
\end{itemize}

\subsubsection{ICMP扫描模块}
\begin{itemize}
    \item \textbf{功能}:实现ICMP ping功能,检测主机可达性
    \item \textbf{输入}:目标IP地址,超时时间
    \item \textbf{输出}:主机可达性状态,RTT时间
    \item \textbf{依赖}:系统网络接口,原始套接字权限
\end{itemize}

\subsubsection{端口扫描模块}
\begin{itemize}
    \item \textbf{功能}:实现多种端口扫描方式
    \item \textbf{输入}:目标地址,端口范围,扫描类型
    \item \textbf{输出}:开放端口列表,扫描统计信息
    \item \textbf{依赖}:网络接口,多线程支持
\end{itemize}

\subsubsection{网络工具模块}
\begin{itemize}
    \item \textbf{功能}:提供底层网络操作支持
    \item \textbf{输入}:网络配置参数
    \item \textbf{输出}:网络操作结果
    \item \textbf{依赖}:系统网络库
\end{itemize}

\section{详细设计}

\subsection{Web前端模块设计}

\subsubsection{模块概述}
Web前端模块负责提供用户界面和交互功能,采用响应式设计,支持桌面和移动设备访问。

\subsubsection{主要数据结构}
\begin{itemize}
    \item \textbf{扫描配置对象}:包含目标地址、端口范围、扫描类型等参数
    \item \textbf{扫描结果对象}:包含开放端口、扫描统计、时间戳等信息
    \item \textbf{历史记录对象}:包含历史扫描的配置和结果
\end{itemize}

\subsubsection{核心函数设计}

\textbf{函数名}:startScan(config)
\begin{itemize}
    \item \textbf{输入参数}:扫描配置对象
    \item \textbf{输出类型}:Promise对象
    \item \textbf{函数功能}:发起扫描请求,处理响应
    \item \textbf{依赖函数}:updateProgress(), displayResults()
    \item \textbf{处理流程}:
    \begin{enumerate}
        \item 验证输入参数
        \item 发送HTTP请求到后端API
        \item 实时更新扫描进度
        \item 接收并处理扫描结果
        \item 更新界面显示
    \end{enumerate}
\end{itemize}

\subsection{后端API模块设计}

\subsubsection{模块概述}
后端API模块基于httplib库构建,提供RESTful风格的HTTP接口,处理前端请求并调用相应的功能模块。

\subsubsection{主要数据结构}
\begin{itemize}
    \item \textbf{请求参数结构}:包含扫描类型、目标地址、端口范围等
    \item \textbf{响应结果结构}:包含状态码、数据内容、错误信息等
    \item \textbf{扫描任务结构}:包含任务ID、状态、进度等信息
\end{itemize}

\subsubsection{核心函数设计}

\textbf{函数名}:handlePortScan(target, scanType, portRange, customPorts, threads, timeout)
\begin{itemize}
    \item \textbf{输入参数}:目标地址、扫描类型、端口范围、自定义端口、线程数、超时时间
    \item \textbf{输出类型}:JSON对象
    \item \textbf{函数功能}:处理端口扫描请求,返回扫描结果
    \item \textbf{依赖函数}:tcpSynScan(), tcpConnectScan(), udpScan()
    \item \textbf{处理流程}:
    \begin{enumerate}
        \item 参数验证和预处理
        \item 根据扫描类型调用相应扫描函数
        \item 收集扫描结果
        \item 格式化返回数据
    \end{enumerate}
\end{itemize}

\subsection{ICMP扫描模块设计}

\subsubsection{模块概述}
ICMP扫描模块实现ping功能,用于检测目标主机的可达性,支持自定义超时时间和重试次数。

\subsubsection{主要数据结构}
\begin{itemize}
    \item \textbf{ICMP头部结构}:包含类型、代码、校验和等字段
    \item \textbf{Ping结果结构}:包含可达性状态、RTT时间、丢包率等
    \item \textbf{网络地址结构}:包含IP地址、端口等信息
\end{itemize}

\subsubsection{核心函数设计}

\textbf{函数名}:ping(target, timeout)
\begin{itemize}
    \item \textbf{输入参数}:目标地址、超时时间
    \item \textbf{输出类型}:std::optional<std::chrono::milliseconds>
    \item \textbf{函数功能}:发送ICMP echo请求,检测主机可达性
    \item \textbf{依赖函数}:createSocket(), sendEchoRequest(), receiveEchoReply()
    \item \textbf{处理流程}:
    \begin{enumerate}
        \item 创建原始套接字
        \item 构造ICMP echo请求包
        \item 发送请求并等待响应
        \item 计算往返时间
        \item 返回结果
    \end{enumerate}
\end{itemize}

\subsection{端口扫描模块设计}

\subsubsection{模块概述}
端口扫描模块实现多种扫描方式,包括TCP SYN扫描、TCP Connect扫描和UDP扫描,支持多线程并发扫描。

\subsubsection{主要数据结构}
\begin{itemize}
    \item \textbf{扫描配置结构}:包含目标地址、端口列表、扫描参数等
    \item \textbf{扫描结果结构}:包含开放端口、过滤端口、统计信息等
    \item \textbf{线程任务结构}:包含线程ID、端口范围、结果容器等
\end{itemize}

\subsubsection{核心函数设计}

\textbf{函数名}:TCPSynScanJson(target, ports)
\begin{itemize}
    \item \textbf{输入参数}:目标地址、端口列表
    \item \textbf{输出类型}:std::vector<int>
    \item \textbf{函数功能}:执行TCP SYN扫描,返回开放端口列表
    \item \textbf{依赖函数}:createRawSocket(), sendSynPacket(), receiveResponse()
    \item \textbf{处理流程}:
    \begin{enumerate}
        \item 创建原始套接字
        \item 构造TCP SYN包
        \item 发送SYN包到目标端口
        \item 监听SYN-ACK响应
        \item 判断端口状态
        \item 返回开放端口列表
    \end{enumerate}
\end{itemize}

\section{系统实现与测试}

\subsection{实现环境}
\begin{itemize}
    \item \textbf{操作系统}:Linux (WSL2 Ubuntu)
    \item \textbf{编译器}:g++
    \item \textbf{构建工具}:CMake, make
    \item \textbf{开发语言}:C++, HTML, CSS, JavaScript
    \item \textbf{主要依赖库}:
    \begin{itemize}
        \item httplib:HTTP服务器库
        \item nlohmann/json:JSON处理库
        \item fmt:字符串格式化库
    \end{itemize}
\end{itemize}

\subsection{测试环境搭建}
测试环境包括:
\begin{itemize}
    \item \textbf{本地测试环境}:WSL2 Ubuntu系统
    \item \textbf{目标测试主机}:本地回环地址、局域网主机
    \item \textbf{网络环境}:局域网环境,支持ICMP和TCP/UDP协议
    \item \textbf{浏览器环境}:Chrome、Firefox、Safari等现代浏览器
\end{itemize}

\subsection{测试方法}
采用以下测试方法:
\begin{itemize}
    \item \textbf{功能测试}:验证各模块功能正确性
    \item \textbf{性能测试}:测试扫描速度和资源占用
    \item \textbf{兼容性测试}:测试不同浏览器兼容性
    \item \textbf{压力测试}:测试高并发扫描性能
    \item \textbf{安全测试}:验证扫描行为的安全性
\end{itemize}

\subsection{测试流程}
测试流程如图\ref{fig:test_flow}所示:

\begin{figure}[h]
\centering
\begin{tikzpicture}[
    box/.style={rectangle, draw, rounded corners, minimum width=2cm, minimum height=1cm, align=center},
    arrow/.style={->, thick}
]
    \node[box] (start) at (0,0) {开始测试};
    \node[box] (env) at (0,-1.5) {环境准备};
    \node[box] (func) at (0,-3) {功能测试};
    \node[box] (perf) at (0,-4.5) {性能测试};
    \node[box] (comp) at (0,-6) {兼容性测试};
    \node[box] (result) at (0,-7.5) {测试结果};
    
    \draw[arrow] (start) -- (env);
    \draw[arrow] (env) -- (func);
    \draw[arrow] (func) -- (perf);
    \draw[arrow] (perf) -- (comp);
    \draw[arrow] (comp) -- (result);
\end{tikzpicture}
\caption{测试流程图}
\label{fig:test_flow}
\end{figure}

\subsection{具体测试内容}

\subsubsection{ICMP扫描测试}
\begin{itemize}
    \item \textbf{测试目标}:本地回环地址(127.0.0.1)、局域网主机
    \item \textbf{测试结果}:成功检测主机可达性,RTT时间准确
    \item \textbf{测试结论}:ICMP扫描功能正常,响应时间在预期范围内
\end{itemize}

\subsubsection{TCP SYN扫描测试}
\begin{itemize}
    \item \textbf{测试目标}:常用端口(80, 443, 22, 21等)
    \item \textbf{测试结果}:准确识别开放端口,扫描速度快
    \item \textbf{测试结论}:SYN扫描功能正常,扫描速度较快
\end{itemize}

\subsubsection{TCP Connect扫描测试}
\begin{itemize}
    \item \textbf{测试目标}:Web服务器、SSH服务器等
    \item \textbf{测试结果}:成功建立连接,准确识别服务状态
    \item \textbf{测试结论}:Connect扫描功能正常,适用于需要完整连接的场景
\end{itemize}

\subsubsection{UDP扫描测试}
\begin{itemize}
    \item \textbf{测试目标}:DNS(53)、DHCP(67,68)等UDP服务
    \item \textbf{测试结果}:能够检测UDP端口状态
    \item \textbf{测试结论}:UDP扫描功能正常,但速度相对较慢
\end{itemize}

\subsubsection{多线程性能测试}
\begin{itemize}
    \item \textbf{测试目标}:不同线程数下的扫描性能
    \item \textbf{测试结果}:线程数增加显著提升扫描速度,但存在最优线程数
    \item \textbf{测试结论}:多线程优化有效,建议线程数设置为100-200
\end{itemize}

\subsubsection{Web界面测试}
\begin{itemize}
    \item \textbf{测试目标}:界面响应性、结果展示、历史记录
    \item \textbf{测试结果}:界面流畅,结果展示清晰,历史记录功能正常
    \item \textbf{测试结论}:Web界面用户体验良好,功能完整
\end{itemize}

\subsection{测试结论}
通过全面测试,系统各项功能均达到预期目标:
\begin{itemize}
    \item \textbf{功能完整性}:所有设计功能均正常实现
    \item \textbf{性能表现}:扫描速度满足实际使用需求
    \item \textbf{稳定性}:系统运行稳定,无明显bug
    \item \textbf{用户体验}:界面友好,操作简单直观
    \item \textbf{兼容性}:支持主流浏览器和操作系统
\end{itemize}

\section{项目总结}

\subsection{项目成果}
本项目成功实现了一个功能完整、性能优良的Web端口扫描工具,主要成果包括:

\begin{itemize}
    \item \textbf{技术实现}:成功集成C++后端和Web前端,实现了完整的端口扫描功能
    \item \textbf{功能特色}:支持多种扫描方式,提供直观的可视化界面
    \item \textbf{性能优化}:通过多线程技术实现了高效的扫描性能
    \item \textbf{用户体验}:提供了友好的Web界面,降低了使用门槛
\end{itemize}

\subsection{技术亮点}
\begin{itemize}
    \item \textbf{架构设计}:采用前后端分离架构,具有良好的可维护性和扩展性
    \item \textbf{多线程优化}:实现了高效的多线程扫描,显著提升扫描速度
    \item \textbf{实时交互}:支持实时进度显示和结果更新
    \item \textbf{响应式设计}:Web界面支持多种设备访问
\end{itemize}

\subsection{项目价值}
\begin{itemize}
    \item \textbf{实用价值}:为网络安全测试提供了实用的工具
    \item \textbf{教育价值}:展示了网络编程和Web开发的综合应用
    \item \textbf{技术价值}:验证了C++和Web技术结合的可能性
\end{itemize}

\subsection{改进方向}
\begin{itemize}
    \item \textbf{功能扩展}:可添加更多扫描方式和服务识别功能
    \item \textbf{性能优化}:可进一步优化扫描算法和并发策略
    \item \textbf{安全增强}:可添加更多安全检测和防护功能
    \item \textbf{用户体验}:可优化界面设计和交互流程
\end{itemize}

\section{分工}

本项目由团队成员共同完成,具体分工如下:

\begin{itemize}
    \item \textbf{项目负责人}:负责项目整体规划和进度管理
    \item \textbf{后端开发}:负责C++后端服务开发和API设计
    \item \textbf{前端开发}:负责Web界面设计和JavaScript开发
    \item \textbf{网络模块开发}:负责ICMP和端口扫描核心功能实现
    \item \textbf{测试验证}:负责系统测试和性能优化
    \item \textbf{文档编写}:负责技术文档和用户手册编写
\end{itemize}


\begin{table}[H]
\centering
\renewcommand{\arraystretch}{1.3} % 可选:增加行高
\setlength{\tabcolsep}{18pt}      % 增加列间距,默认6pt,可根据需要调整
\begin{tabular}{|p{3cm}|p{2cm}|p{5cm}|p{3cm}|}
\hline
姓名 & 是否组长 & 任务 & 评分 \\
\hline
刘梓芃 & 是 & 内容A & 说明A \\
\hline
聂鸣涛 & 否 & 内容B & 说明B \\
\hline
李卓恒 & 否 & 内容C & 说明C \\
\hline
张煜哲 & 否 & 内容D & 说明D \\
\hline
\end{tabular}
\caption{项目组成员贡献表}
\end{table}
%%----------- 参考文献 -------------------%%
%在reference.bib文件中填写参考文献,此处自动生成

\reference


\end{document}